\documentclass{article}

%
% packages
%
% references
\usepackage[backend=biber]{biblatex}
% better padding
\usepackage[margin=100px]{geometry}
% images
\usepackage{graphicx}
% math symbols
\usepackage{amsmath}
% more symbols
\usepackage{textcomp}
% more math symbols
\usepackage{amssymb}
% links
\usepackage{hyperref}
% linsk jump to top of figures
\usepackage[all]{hypcap}
% floating options ([H])
\usepackage{float}
% source code
\usepackage{listings}
% indent first paragraph after section
\usepackage{indentfirst}
% continous enumeration
\usepackage{enumitem}
% cmark and xmark
\usepackage{pifont}
% colors
\usepackage[table,x11names]{xcolor}
% column style
\usepackage{tabu}
% section settings
\usepackage{titlesec}

\addbibresource{~/scripts/latex/ref.bib}

%
% custom commands
%
\newcommand{\N}{\mathbb{N}}
\newcommand{\lxor}{\veebar}
\newcommand{\cmark}{\ding{51}}
\newcommand{\xmark}{\ding{55}}

% path for images
\graphicspath{{images/}}

% link stuff
\hypersetup{
	colorlinks=true,
	linkcolor=blue,
	filecolor=red,
	runcolor=red,
	urlcolor=cyan,
	anchorcolor=black,
	citecolor=.,
	menucolor=red,
	pdftitle={OR STM32H7 App Poro\v{c}ilo},
}

% Extra subsections (parargaph)
\setcounter{secnumdepth}{4}

\titleformat{\paragraph}
{\normalfont\normalsize\bfseries}{\theparagraph}{1em}{}
\titlespacing*{\paragraph}
{0pt}{3.25ex plus 1ex minus .2ex}{1.5ex plus .2ex}

% programming stuff
\lstset{language=Python}


\author{}
\title{OR STM32H7 App Poro\v{c}ilo}
\date{2023}

\begin{document}

\maketitle

\tableofcontents

\clearpage


\section{Delovanje}
\noindent
Program deluje tako, da iz senzorja svetlobe (ki je priklopljen na pin
A1) vsake 0.5 sekunde prebere vrednost. \v{C}e zazna premajhno
koli\v{c}ino svetlobe pri\v{z}ge rumene lu\v{c}i (LED, ki so priklopljene
na pin D4). Koli\v{c}ino svetlobe tudi sporo\v{c}i preko serjiske
komunkacije. Tukaj je veliko mo\v{z}nosti za nadgradnjo z ne
preve\v{c} dela, npr lahko bi senzor svetlobe postavili zunaj, LED pa
bi iz digitalnega signala spremenili v analognega in bi imeli sistem
za simulacijo zunanje svetlobe (lu\v{c}i se zjutraj zatemnijo, da
simulirajo son\v{c}ni vzhod). Lahko pa bi namesto senzorja svetlobe
priklopili senzor gibanja in bi imeli avtomatizirane lu\v{c}i. \\

\noindent
Program pa vsebuje tudi senzor teko\v{c}ine (priklopjen na D7), ki
ob zaznanju teko\v{c}ine sprozi prekinitev, katera ugasne rumene
lu\v{c}i in pri\v{z}ge ``alarm'' (utripanje modrih lu\v{c}i), ki je
spro\v{z}en dokler senzor zaznava teko\v{c}ino. Preko serjiske
komunikacije sporo\v{c}i, ko zazna teko\v{c}ino in ko jo neha
zaznavati.

\section{Vezje}
\noindent
Vezje je na sliki spodaj, le da je namesto senzorja vlage v tleh
uporabljen sensor teko\v{c}ine.

\begin{figure}[H]
	\begin{center}
	  \includegraphics[scale=0.6]{scheme.png}
	\end{center}
	\caption{Vezje}
\end{figure}

\section{Program}
\noindent
Programiral sem v programskem jeziku C, v CubeIDE. Uporabil sem
header datoteke, ki jih izdaja ST
(\href{https://www.st.com/en/embedded-software/stm32cubeh7.html}
{https://www.st.com/en/embedded-software/stm32cubeh7.html}), nahajajo
se v mapi chip\_headers. \\
Svoj program se poskusal lo\v{c}iti na \v{c}im majn\v{s}e dele, ki
spadajo skupaj (ADC, UART, SYSTICK, EXTI, FPU) in vsakega dal v svojo
datotko. Celoten projekt je tudi objavlen na github:
\href{https://github.com/tibozic/OR\_STM32H750\_sensor\_app}
{https://github.com/tibozic/OR\_STM32H750\_sensor\_app}




\end{document}
